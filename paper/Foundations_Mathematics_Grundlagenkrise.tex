\documentclass[hidelinks]{article}
\usepackage[english]{babel} 
\usepackage[utf8x]{inputenc}
%% Hyperlinks 
\usepackage{hyperref}
\hypersetup{
    colorlinks,
    linkcolor={red!50!black},
    citecolor={blue!50!black}, linktoc=all,
    urlcolor={blue!80!black}
}
%% Graphics
\usepackage{graphicx}
\usepackage{float}

\usepackage{enumerate}
% Math packages
\usepackage{amsmath}
\usepackage{amssymb}

% Proof system
\usepackage{amsthm}
\theoremstyle{plain}
\newtheorem{thm}{Theorem}[section]
\newtheorem{lem}[thm]{Lemma}
\newtheorem{prop}[thm]{Proposition}
\theoremstyle{definition}
\newtheorem{defn}[thm]{Definition}
\newtheorem{exmpl}[thm]{Example}
\newtheoremstyle{rem} % name
    {\topsep}                    % Space above
    {\topsep}                    % Space below
    {}                   % Body font
    {}                           % Indent amount
    {\bf}                   % Theorem head font
    {:}                          % Punctuation after theorem head
    {.5em}                       % Space after theorem head
    {}  % Theorem head spec (can be left empty, meaning ‘normal’)
\theoremstyle{rem}
\newtheorem*{remark}{Note}
%\usepackage{xpatch}
\makeatletter
%% Remove last point from definitions, theorems, etc.
%\xpatchcmd{\@thm}{\thm@headpunct{.}}{\thm@headpunct{\\}}{}{}
%\makeatother

% paper margins
\usepackage[margin=1.5in]{geometry}
% citations
\usepackage{cite}
% Graphs
\usepackage{tikz}
\usetikzlibrary{calc,arrows.meta,positioning}
\usepackage{tikz-3dplot}
\usepackage{subfig}
\usepackage{pgfplots}
\pgfplotsset{%
    ,compat=1.12
    ,every axis x label/.style={at={(current axis.right of origin)},anchor=north west}
    ,every axis y label/.style={at={(current axis.above origin)},anchor=north east}
    }

% Custom commands
\newcommand{\mx}{\mathcal{X}}
\newcommand{\fromto}[2]{\{#1,\ldots,#2\}}

\pagestyle{plain}

%Title page settings
\usepackage[affil-it]{authblk}

% Title of document
\title{\textbf{Foundations of Mathematics - Grundlagenkrise}}
% Author
\author{Kevin Kappelmann}
\affil{Chair for Logic and Verification,\\ Technical University of Munich}
\date{\today}


% Quotes with author
\def\signed #1{{\leavevmode\unskip\nobreak\hfill\penalty50\hskip2em
  \hbox{}\nobreak\\\hfill \raggedleft--- #1%
  \parfillskip=0pt \finalhyphendemerits=0 \endgraf}}

\newsavebox\mybox
\newenvironment{aquote}[1]
  {\savebox\mybox{#1}\begin{quote}}
  {\signed{\usebox\mybox}\end{quote}}

% Line breaks after paragraph
\usepackage[parfill]{parskip}
%------------------------------------------------------------------------------
\begin{document}

\pagenumbering{gobble}

\maketitle
\newpage
\section*{Preface}
Although --- or maybe even because --- mathematics is likely the most well-conceived and exact science of mankind, it has not been free of scepticism and controversies. In its heart, mathematics is dealing with the discovery of unchangeable, sempiternal truths. For this, we are using rigorous proofs. But what is it that we call a proof? 

One might define a proof as a coherent chain of logical arguments leading from a set of premises to a conclusion. This definition, however, raises many new questions. What deserves to be called coherent? When is an argument logical? Do all humans follow the same rules of logic? Which premises are there to begin with?

Some of these difficult questions were essential during a heated phase in the beginning of the 20th century: the foundational crisis of mathematics. In this period, some of the greatest mathematicians tried to give different explanations for many of named questions. As a result of longstanding debates and rigorous work, we received the sophisticated foundations of modern mathematics.

Owing to this achievements, contemporary mathematicians are able to concentrate on the extension of mathematics rather than paying attention to its foundations. However, due to the invention of the computer in the late 20th century, a new field of proof theory has arisen. Computer scientists and mathematicians began to develop proof assistants and automated theorem provers. While former is designed to verify proofs typed by humans, latter acts on a fully automatic basis.

In particular automatic theorem provers gave raise to new questions regarding mathematical proofs. Does a computer generated proof have the same credibility as a proof written by a person? What is a proof and how can I persuade somebody that I am right? Does it suffice to understand every isolated step of it, or do I need to understand its holistic idea? Must it be accepted by some, a few, or even just one person? It seems like we are facing a new foundational crisis.

Nevertheless, these controversies are not focused by this paper, but rather they are subjects for debate in the seminar ``Formal Proof in Mathematics and Computer Science'' offered by the Chair for Logic and Verification at the Technical University of Munich in 2017. This paper shall start off the seminar by giving a brief overview of the foundations of mathematics with a focus on the foundational crisis in the 20th century.
\newpage

\tableofcontents 
\listoffigures 
\listoftables 
\newpage

\pagenumbering{arabic}
\section{Historical introduction}
While one might think that mathematics as a subject about absolute certainty ought to be free of controversies, history taught us otherwise. In fact, it has been subjected to fiery disputations since ancient greece. Before diving into the most famous controversy in the history of mathematics, namely the foundational crisis (in German \textit{Grundlagenkrise}), we want to give a brief historical overview focusing on the foundations and some of the most well-known controversies\footnote{We emphasise the smaller extent and impact of various mathematical disputations by using the word \textit{controversy} as opposed to \textit{crisis}.} of mathematics.

\subsection{Ancient mathematics}

The history of mathematics is coined by a series of abstractions.
Add two apples to three other apples, and you get five apples. Add two bananas to three other bananas, and you get five bananas. We abstract our results and derive: two plus three equals five. The concept of numbers might be the first mathematical abstraction in history. Unsurprisingly, arithmetics and geometry - owing to their inuitive nature - were the first developed branches of mathematics.

First evidence for more complex mathematics dates back to around 2400 BC\@. Egyptians and Babylonians used basic arithmetics and geometry for trading, taxation, building and construction, and land measurement. Mathematics by this time, however, was just regarded as an applied tool to solve practical problems rather than an exact science. Heuristics, pictures and vague analogies justified the use of given formulas rather than rigorous proofs.

The idea of demonstrating conclusions using coherent arguments, and thereby the development of the notion of a proof, was one of the great achievements in ancient Greece.
%In fact, ancient Greek mathematicians accomplished an intriguing progress with . 
Beginning with Thales in 600 BC, mathematics as a means for the exploration of perpetual truths began to raise. To no surprise, proofs where focused on arithmetics and geometry, which both seemed to consist of unquestionable truths.

In 500 BC the Pythagoreans did not only invent one of the most famous theorems in mathematic, but to our larger interest experienced the first noteworthy mathematical controversy. They were in firm belief that all numbers were commensurable; that is, for every pair of non-zero numbers $a$ and $b$, the ratio $\frac{a}{b}$ is a rational number. It allegedly was the work of Hippasus who dismissed this ideal by proofing the existence of irrational numbers. Legend has it that Hippasus was sentenced to death by drowning for the discovery of this unbearable truth. What remains as a fact, however, is that Pythagorean mathematics changed drastically after this discovery.

Meanwhile, the Greek philosopher Zeno of Elea devised four paradoxes, now commonly known as Zeno's paradoxes, which mainly deal with the illusion of motion. The first paradox, referred to as Achilles and the tortoise, can be recounted as follows:
\begin{quote}\label{zeno_paradox}
Achilles and the Tortoise want to conduct a race. Achilles gives the Tortoise a head start since he clearly is the faster runner. But by the time Achilles has reached the point where the Tortoise started, the slow individual will have moved on a few steps to a new position. When Achilles again reaches this new position, the labouring Tortoise will have moved on again. Each time Achilles reaches the point where the Tortoise was, the cunning reptile will always have moved a little way ahead; hence, it will always hold a lead.
\end{quote}
Readers that took a foundational calculus class may already be able to refute this paradox. The solution demands for the calculus of infinitesimals and the concept of the continuum chiefly developed in the 17th century. Up to that time, no man was able to give Zeno a satisfying answer.

We want to forestall that paradoxes indeed turned out to not only pose a fruitful, philosophical subject of conversation but also play an important part in the history of mathematics. By questioning the truth of the seemingly indisputable, they offer the possibilty to reveal the inconsistency of a mathematical system at its core. We shall say more about this later.

Finally, we ought to name the most influential mathematical work of ancient Greece if not even of all time\footnote{Up to the middle of the 19th century, Euclid's Elements is said to had been having a wide circulation rivaled only by the bible.}: Euclid's Elements. The collection, consisting of 13 books, was written around 300 BC by the Greek mathematician Euclid in Alexandria. The books cover Euclidian geometry (``the geometry of our world'') as well as elementary number theory, albeit not in an algebraic but geometric way. The intriguing thing, besides its exceptional volume, is its axiomatic, deductive treatment of mathematics. To this day, mathematicians most commonly use systems axiomatised in the same fashion as Euclid did over 2000 thousand years ago. Reading a proof from Euclid's Elements feels suprisingly similar to reading a proof from a modern textbook.
\begin{aquote}{Bertrand Russell}
	``At the age of eleven, I began Euclid, with my brother as tutor. This was one of the great events of my life, as dazzling as first love. I had not imagined there was anything so delicious in the world.''\footnote{The Autobiography of Bertrand Russell, Chapter 1: Childhood, pp. 30--31}
\end{aquote}
Euclid's Elements was manifested as a work of timeless certainty. Nobody could doubt that! Or at least, so had been thought for a long time. The discovery of non-Euclidian geometry, though not shaking the consistency of Euclid's work, questioned the exclusive existence of a geometry as postulated by Euclid. We will discuss this in more detail in section~\ref{ssec_causes}.

\subsection{An infinitely small crisis}
The Greeks, albeit discussing the possibility of the continuum, conducted a ``static'' way of mathematics in the sense that subjects of interest were mainly those that deal with stationary objects, e.g.\ arithmetic and geometry. In addition, it was widely belived that objects, in particular time and space, were only finitely divisible. This kind of mathematics, however, is incapable of finding adequate answers to Zeno's paradoxes (see section~\ref{zeno_paradox}).
It was not until the middle of the 17th century that Leibniz as well as Newton independently developed the tools for infinitesimal calculus capable of solving the mistery of infinite divisibilty.

Infinitesimal calculus, these days simply known as calculus, is the study of continuous change. Although Leibniz and Newton were able to calculate the derivatives and integrals of functions using a notion of ``infinite small quantities'', they were not able to deliver an elaborated foundation for their used methods. Rather, they used heuristic principles, such as the law of continuity\footnote{the principle that "whatever succeeds for the finite, also succeeds for the infinite"}, as justification which gave raise to valid scepticism and critques most notably by Bishop Berkeley in 1734. He addressed the uncertainty and ominosity of the calculus derived by Leibniz and Newton in his book ``The Analyst''. For this, he satirically compared the use of ``infinite small quantities'' and its vague justification with critiques common to religion. His book subtitled:
\begin{quote}
	``A DISCOURSE Addressed to an Infidel MATHEMATICIAN\@. WHEREIN It is examined whether the Object, Principles, and Inferences of the modern Analysis are more distinctly conceived, or more evidently deduced, than Religious Mysteries and Points of Faith''
\end{quote}
In addition to this uncertainty, when Newton and Leibniz first published their results, there was great controversy over which mathematician deserved credit. These factors ultimately caused what we shall call the second controversy of mathematics which divided English-speaking mathematicians from continental European mathematicians for many years\footnote{Despite its critiques, infinitesimal calculus kept being used as a successful means of calculation. Its sound foundations were formalised 150 years later by Cauchy and Weierstrass.}.

\section{The foundational crisis}
While controversies like that of the Pythagoreans or the infinitesimal calculus certainly influenced the development of modern mathematics, they do not deserve to be called a proper crisis, since they were restricted to a small tract of mathematics or did not cause an exceptional impact on mathematical foundations.

The foundational crisis of mathematics started in 1902 with Russell's antinomy and ended in 1931 with an astonishing twist by Kurt Gödel and his incompleteness theorems.
We will firstly examine the causes that lead to this milestone of mathematics and then proceed to chronologically discuss involved persons, their representated movements, and important events by providing philosophical as well as mathematical background and details.

\subsection{Causes}\label{ssec_causes}
In 19th century, Euclid's Elements had been established as a mathematical showcase for more than 2000 years due to its axiomatic, deductive treatment of mathematics. The first book began with his five postulates\footnote{Euclid used the word \textit{postulate} instead of \textit{axiom}.} of geometry:
\begin{quote}
``Let the following be postulated:
\begin{enumerate}
\item To draw a straight line from any point to any point.
\item To produce [extend] a finite straight line continuously in a straight line.
\item To describe a circle with any centre and distance [radius].
\item That all right angles are equal to one another.
\item That, if a straight line falling on two straight lines make the interior angles on the same side less than two right angles, the two straight lines, if produced indefinitely, meet on that side on which are the angles less than the two right angles.''
\end{enumerate}
\end{quote}
While postulates 1--4 seem fairly understandable and easy, many mathematicians became curious about Euclid's fifth postulate. The postulate, which to this day is known as the parallel postulate, seemed fairly more complicated than its four predecessors and troubled many geometers for at least a thousand years. Many believed it could be proved as a theorem from the first four postulates, but all attempts failed. 

In circa 1813, Carl Friedrich Gauß worked out that in fact the fifth postulate is independent of the other postulates; that is to say, the parallel postulate as well as its negation can be added to the first four postulates without causing any inconsistencies. While the first case gives us the geometry as postulated by Euclid, the latter revealed a new kind of non-Euclidian geometry.

Though Gauß did not publish his discovery, it was rediscovered just a few years later. The announcement had a large impact. Suddenly, Euclid's Elements, and with it, the certainty of mathematics, began to totter; the developed axiomised systems were under fire --- which axioms pose the truth, which will cause harm?

What followed is a series of rigorous axiomatisation of known systems. While Peano began to build a foundation for arithmetics, Pasch and Hilbert modernised the foundations of geometry. 

Peano axiomatised arithmetic of natural numbers and Moritz Pasch and Hilbert Geometry (modern form); while Gottlob Frege tried to build foundation solely using logic for mathematics (no math.\ symbols) (``Grundgesetze der Arithmetik'') -> Logicism. But inconsistencies destroyed dreams (Russel antinomy 1902) -> Frege stopped its undertakings.
Many did not pay much attention to that problem but those who were interested understood its difficulty (Cantors set theory as a foundation of mathematics was also affected).
The set of all sets that do not contain itself was not contradictory to Cantor's set theory but with logic proofed by Russell -> better to use axiomses instead of relying on regarding mathematical objects as immaterial truths without limitation and reflections.\\

Three schools: Logicism, Intuitionism, Formalism\\

Logicism:
Russel and Whitehead. Principia Mathematica (1910-1913) mathematics was an extension of logic. Try to avoid antinomies by using fine grained incremental steps (type theory of Russell). But steps were widely critised as they did not seem to be logical evident but contained an ontological claim of validity (axiom of infinity, reducibility axiom). Hence, the movement failed.\\

Formalism:\\
Supported the autonomy of mathematics in contrast to logicism. Math is based on symbols and will be conducted by using axioms. It does not have to justify the existence of its objects, but its use ist justified by its success. Only restriction: Operations are not allowed to lead to a cotradiction -> proof system consistent. This needs a new field of studies: meta-mathematic. Hilbertprogram ist necessary (justfied by anomalies like Russel's). Hilbert axiomatised geometry, Zermelo axiomatised Cantor's set theory (which was not regarded as a foundation of mathematics at first as Zermelo said that its consistency proof is a distant prospect; after extensions from Fraenkel and Skolem this changed and replaced Princ.\ Math.\ as foundation).  \\

Intutionism:\\
Math is not just made of symbols, and it is not existent in a world idependent of mankind, but it is only existent in humans' minds produced by intuition. That creation process is neither of a symbolic kind nor is it reducible to pure logic. Symbols and logic are merely representations used for these mental creations. Formalism is a void game without meaning and mathematical operations (transfinite set theory? Ugh\ldots potential vs.\ actual infinities).\\
Important restriction: No law of excluded middle! (use example \url{https://en.wikipedia.org/wiki/Law\_of\_excluded\_middle#example}). Main advocate was Brouwer who developed a set theory without named law. (more accepted after world war??)\\

In 1922 Weyl published a paper (after meeting Brouwer) which stated ``Brouwer - this is revolution''. Weyl was a student of Hilbert. Hilbert was not amused and compared Brouwer's and Weyl's intentions as an attempted coup that will fail. Hilbert then resumed his meta-mathematical work which he had paused.\\

The following years were characterised by many articles which spread the dispute between the both schools of thought. Intuitionism was hard to define (spreaded across many papers from Brouwer and written in dutch) and formalism was not fully developed yet but still under development. Hilbert's followers were extremly active and gained many supporters. 1925 proof by Ackermann (student of Hilbert) about the consistency of law of excluded middle. Confidence in formalism grew. Meanwhile, criticism for Intuitionism grew as its limitations would harm the applicability of mathematics. Even Weyl agreed to that in 1924.\\

In 1928 German mathematicans were allowed to attend the international congress for the first time after WW1. However, they were not allowed to vote and Bouwer solidarised with them and called on the Germans to boycott the congress. Nevertheless, most Germans attended the congress including Hilbert. He presented his foundational programm and Bouwer was not able to discount Hilbert as he did not attended the congress. A few days afterwards, Hilbert as one of the major publishers of the prestigous mathematical magazine ``Mathematischen Annalen'' decided to exclude Brouwer as a co-publisher without conducting a referendum. This lead to confusion and disputes but in the end Hilbert's wish came true and Brouwer was excluded. After this incident, Brouwer stopped publishing articles dealing with Intuitionism. As a result, intutionism lost its public attention. \\

Formalism seemed to be the winner. Goedel's incompleteness theorem destroyed the hope of a proofed consistent formal system. Nevertheless, modern mathematics is build on formalism. What axioms do I need at least in order to proof this theorem are current questions of formalism. The justification of the axioms are often regarded philosophical work and are often not regarded by mathematicians.
\\

Topic overview:\\
At the beginning of the 20th century it became clear that mathematics needed formal foundations, which was partly caused by discovering various paradoxes. A couple of solutions were proposed including Hilbert's program of formalism. The crisis was partially solved by adopting the new foundations of set theory although Hilbert's ambitions were not completely fulfilled due to Gödel. The student should cover main events and players from this story and accompany the presentation by the most important technical details.



Quotes: 
\begin{itemize}
\item „Aus dem Paradies, das Cantor uns geschaffen, soll uns niemand vertreiben können.“ -– David Hilbert: Über das Unendliche, Mathematische Annalen 95 (1926), S. 170 
\item Dieses Tertium non datur (satz ausgeschlossenem dritten) dem Mathematiker zu nehmen, waere etwa, wie wenn man dem Astronomen das Fernrohr oder dem Boxer den Gebrauch der Faeuste untersagen wollte.“ – David Hilbert: Die Grundlagen der Mathematik, Abhandlungen aus dem mathematischen Seminar der Hamburgischen Universität, 6. Band (1928), S. 80
\item Theory, by Hrbacek and Jech, we read on page 54 :
	"Axiom of Infinity. An inductive (i.e.\ infinite) set exists." Compare this against the axiom of God as presented by Maimonides (Mishneh Torah, Book 1, Chapter 1): The basic principle of all basic principles and the pillar of all the sciences is to realize that there is a First Being who brought every existing thing into being.  Mathematical axioms have the reputation of being self- math experience page 171
\end{itemize}

\newpage
\bibliographystyle{plain}
\bibliography{sources}
\end{document}

%%%%%%%%%%%%%%%%%%%%%%%%%%%%%%%%%%%%%%%%%
% Beamer Presentation
% LaTeX Template
% Version 1.0 (10/11/12)
%
% This template has been downloaded from:
% http://www.LaTeXTemplates.com
%
% License:
% CC BY-NC-SA 3.0 (http://creativecommons.org/licenses/by-nc-sa/3.0/)
%
%%%%%%%%%%%%%%%%%%%%%%%%%%%%%%%%%%%%%%%%%

%----------------------------------------------------------------------------------------
%	PACKAGES AND THEMES
%----------------------------------------------------------------------------------------

\documentclass{beamer}

\mode<presentation>{% The Beamer class comes with a number of default slide themes
% which change the colors and layouts of slides. Below this is a list
% of all the themes, uncomment each in turn to see what they look like.

%\usetheme{default}
%\usetheme{AnnArbor}
%\usetheme{Antibes}
%\usetheme{Bergen}
%\usetheme{Berkeley}
\usetheme[compress]{Berlin}
%\usetheme{Boadilla}
%\usetheme{CambridgeUS}
%\usetheme{Copenhagen}
%\usetheme{Darmstadt}
%\usetheme{Dresden}
%\usetheme{Frankfurt}
%\usetheme{Goettingen}
%\usetheme{Hannover}
%\usetheme{Ilmenau}
%\usetheme{JuanLesPins}
%\usetheme{Luebeck}
%\usetheme{Madrid}
%\usetheme{Malmoe}
%\usetheme{Marburg}
%\usetheme{Montpellier}
%\usetheme{PaloAlto}
%\usetheme{Pittsburgh}
%\usetheme{Rochester}
%\usetheme{Singapore}
%\usetheme{Szeged}
%\usetheme{Warsaw}

% As well as themes, the Beamer class has a number of color themes
% for any slide theme. Uncomment each of these in turn to see how it
% changes the colors of your current slide theme.

%\usecolortheme{albatross}
%\usecolortheme{beaver}
%\usecolortheme{beetle}
%\usecolortheme{crane}
%\usecolortheme{dolphin}
%\usecolortheme{dove}
%\usecolortheme{fly}
%\usecolortheme{lily}
%\usecolortheme{orchid}
%\usecolortheme{rose}
%\usecolortheme{seagull}
%\usecolortheme{seahorse}
%\usecolortheme{whale}
%\usecolortheme{wolverine}

\makeatother
\setbeamertemplate{footline}
{
  \leavevmode%
  \hbox{%
  \begin{beamercolorbox}[wd=.3\paperwidth,ht=2.25ex,dp=1ex,center]{section in head/foot}%
    \usebeamerfont{author in head/foot}\insertshortauthor
  \end{beamercolorbox}%
  \begin{beamercolorbox}[wd=.6\paperwidth,ht=2.25ex,dp=1ex,center]{subsection in head/foot}%
    \usebeamerfont{title in head/foot}\insertshorttitle\hspace*{3em}
  \end{beamercolorbox}%
  \begin{beamercolorbox}[wd=.1\paperwidth,ht=2.25ex,dp=1ex,center]{title}%
    \usebeamerfont{title in head/foot}\insertshortinstitute \hspace*{1ex}
  \end{beamercolorbox}}%
  \vskip0pt%
}
\makeatletter

%\setbeamertemplate{footline} % To remove the footer line in all slides uncomment this line
%\setbeamertemplate{footline}[page number] % To replace the footer line in all slides with a simple slide count uncomment this line

\setbeamertemplate{navigation symbols}{} % To remove the navigation symbols from the bottom of all slides uncomment this line
}

\usepackage[english]{babel} 
\usepackage[utf8x]{inputenc}

\usepackage{graphicx} % Allows including images
\usepackage{booktabs} % Allows the use of \toprule, \midrule and \bottomrule in tables
\usepackage{multicol} 

% Math packages
\usepackage{amsmath}
\usepackage{mathtools}
\usepackage{amssymb}
\usepackage{mathpartir}

%----------------------------------------------------------------------------------------
%	TITLE PAGE
%----------------------------------------------------------------------------------------

\title[Mathematical Foundations and Crises]{Foundations of Mathematics\\and the Foundational Crisis} % The short title appears at the bottom of every slide, the full title is only on the title page

\author{Kevin Kappelmann} % Your name
\institute[TUM] % Your institution as it will appear on the bottom of every slide, may be shorthand to save space
{Technical University of Munich}
\date{\today} % Date, can be changed to a custom date

\begin{document}

\begin{frame}
\titlepage % Print the title page as the first slide
\end{frame}

%----------------------------------------------------------------------------------------
%	PRESENTATION SLIDES
%----------------------------------------------------------------------------------------
\section*{Introduction}
%------------------------------------------------

\begin{frame}
    \frametitle{}
    \begin{itemize}[<+->]
	\item What is the most exact science of humanity?
	\begin{itemize}
		\item Well, obviously it is math, right?
	\end{itemize}
	\item I shall convince you with rigorous proofs!
	\begin{itemize}
		\item What is a proof?
		\item What is logical?
	\end{itemize}
	\item Can we trust each other?
	\item Can we trust a computer?
    \end{itemize}
\end{frame}

\begin{frame}
    \frametitle{}
    \begin{itemize}[<+->]
	\item When do I accept a proof?
	\begin{itemize}
		\item Do I need to understand every isolated step?
		\item Do I need to understand its holistic idea?
	\end{itemize}
	\item Computer-generated proofs are barely verifiable by hand.
	\begin{itemize}
		\item[$\Rightarrow$] Controversies
	\end{itemize}
	\item But it is not the first controversy in mathematical history\ldots
    \end{itemize}
\end{frame}


%------------------------------------------------
% Overview slide
%------------------------------------------------

\begin{frame}
    \frametitle{Overview} 
    %\begin{multicols}{2}
        \tableofcontents
    %\end{multicols}
\end{frame}

%------------------------------------------------
\section{History of Mathematics}
%------------------------------------------------
\subsection{Ancient Mathematics}
\begin{frame}
    \frametitle{1+1=2}
    \begin{itemize}[<+->]
	\item How did it begin?
	\begin{itemize}
		\item One apple and another apple are two apples.
		\item One banana and another banana are two bananas.
		\item[$\Rightarrow$] One plus one is two!
	\end{itemize}
	\item Math is about abstraction.
    \end{itemize}
\end{frame}
%------------------------------------------------
\subsection{Infinitesimal Controversy}
\begin{frame}
    \frametitle{I can draw you the world}
    \begin{itemize}[<+->]
	\item What also seems quite intuitive?
	\begin{itemize}
		\item Geometry!
	\end{itemize}
	\item Arithmetic and geometry were the first branches of mathematics.
    \end{itemize}
\end{frame}
%------------------------------------------------
\section{Foundational Crisis}
%------------------------------------------------
\subsection{Causes}
\begin{frame}
    \frametitle{I Think I Broke Math}
\end{frame}
%------------------------------------------------
\subsection{Logicism}
\begin{frame}
    \frametitle{}
\end{frame}
%------------------------------------------------
\subsection{Formalism}
\begin{frame}
    \frametitle{}
\end{frame}
%------------------------------------------------
\subsection{Intuitionism}
\begin{frame}
    \frametitle{}
\end{frame}
%------------------------------------------------
\subsection{Peak and End}
\begin{frame}
    \frametitle{}
\end{frame}
%------------------------------------------------
\section{Aftermath and Prospects}
\begin{frame}
    \frametitle{}
\end{frame}
%------------------------------------------------
\section*{}
%------------------------------------------------
\begin{frame}
    \Huge{\centerline{The End}}
\end{frame}
%----------------------------------------------------------------------------------------

\end{document} 
